%!TEX root = ../main.tex

\begin{refsection}

\section{Quantum Fourier transform}\label{prim:QFT}

\subsubsection*{Rough overview (in words)}

The quantum Fourier transform (QFT) is a quantum version of the discrete Fourier transform (DFT) and takes quantum states to their Fourier transformed version.

%%%%%%%%%%%%%%%%%%%%%%%%%%%%%%%%%%%%%%%%%%%%%%%%%%%%%%%%%%%%%%%%%%%%%%

\subsubsection*{Rough overview (in math)}

The QFT is a quantum circuit that takes pure $N$-dimensional quantum states $\ket{x}=\sum_{i=0}^{N-1}x_i\ket{i}$ to pure quantum states $\ket{y}=\sum_{i=0}^{N-1}y_i\ket{i}$ with the Fourier transformed amplitudes
\begin{align}\label{eq:Fourier}
y_k=\frac{1}{\sqrt{N}}\sum_{l=0}^{N-1}x_l\exp(2\pi ikl/N)\quad\text{for }k=0,\cdots,N-1.
\end{align}

%%%%%%%%%%%%%%%%%%%%%%%%%%%%%%%%%%%%%%%%%%%%%%%%%%%%%%%%%%%%%%%%%%%%%%

\subsubsection*{Dominant resource cost (gates/qubits)}

The space cost is $\bigO{\log(N)}$ qubits and the quantum complexity of the textbook algorithm is $\bigO{\log^2(N)}$. In terms of Hadamard gates, swap gates, and controlled phase shift gates $\ket{0}\bra{0}\otimes I + \ket{1}\bra{1}\otimes R_\ell$ with
\begin{align}
R_\ell=\begin{pmatrix} 1 & 0\\ 0 & \exp\left(2\pi i2^{-\ell}\right)\end{pmatrix}\,,
\end{align}
the quantum circuit looks as follows \cite[Fig.~5.1]{nielsen2002QCQI}, where $N=2^n$:
\vspace{3mm}
\[
\begin{array}{c}
\Qcircuit @C=0.5em @R=1em {
  & \gate{H} & \gate{R_2} & \qw && \cdots && & \gate{R_{n-1}} & \gate{R_n} & \qw & \qw && \cdots && & \qw & \qw & \qw && \cdots && & \qw & \qw & \qw & \qw & \qswap & \qw & \qw && \cdots \\
  & \qw & \ctrl{-1} & \qw && \cdots && & \qw & \qw & \gate{H} & \qw && \cdots && & \gate{R_{n-2}} & \gate{R_{n-1}} & \qw && \cdots && & \qw & \qw & \qw & \qw & \qw & \qswap & \qw && \cdots \\
  & & &&& \vdots & &&& & & & & \vdots & & & & & & & \vdots & & & & & & & & & & & &&&&&&& \\
  % \\
  & \qw & \qw & \qw && \cdots && & \ctrl{-3} & \qw & \qw & \qw && \cdots && & \ctrl{-2} & \qw & \qw && \cdots && & \gate{H} & \gate{R_2} & \qw & \qw & \qw & \qswap \qwx[-2] & \qw && \cdots \\
  & \qw & \qw & \qw && \cdots && & \qw & \ctrl{-4} & \qw & \qw && \cdots&&  & \qw & \ctrl{-3} & \qw && \cdots && & \qw & \ctrl{-1} & \gate{H} & \qw & \qswap \qwx[-4] & \qw & \qw && \cdots
  }
\end{array}
\]
\vspace{3mm}
The swap gates at the end of the circuit are required to reverse the order of the output bits. 
The complexity can be improved to
\begin{align}
\bigO{\log(N)\log\left(\log(N)\epsilon^{-1}\right)+\log^2\left(\epsilon^{-1}\right)}
\end{align}
when only asking for $\epsilon$-approximate solutions \cite{hales2000ImprovQFT}. Finite constants and compilation cost for fault-tolerant quantum architectures are also discussed in the literature. For example \cite{nam2020ApproxQFTgates} gives an implementation with $\bigO{\log(N)\log\log(N)}$ $T$-gates and estimates finite $T$-gate costs for different instance sizes.

%%%%%%%%%%%%%%%%%%%%%%%%%%%%%%%%%%%%%%%%%%%%%%%%%%%%%%%%%%%%%%%%%%%%%%

\subsubsection*{Caveats}

\begin{itemize}
    \item The QFT does not achieve the same task as the classical DFT that takes vectors $(x_0,\cdots,x_{N-1})\in\mathbb{C}^N$ to vectors $(y_0,\cdots,y_{N-1})\in\mathbb{C}^N$ with $y_k$ defined as in Eq.~\eqref{eq:Fourier}. The DFT can be implemented via the fast Fourier transform in classical complexity $\bigO{N\log(N)}$, which is exponentially more costly than the quantum complexity $\bigO{\log^2(N)}$ of the QFT. However, for the QFT to achieve the same task as the DFT, pure state quantum \hyperref[prim:Tomography]{tomography} would be required to read out and learn the Fourier-transformed amplitudes, which destroys any quantum speedup for the DFT.
    \item When QFT is employed in use cases, e.g., for factoring, one has to be careful in finite size instances when counting resources \cite{smolin2013OverQF}, and for this a semi-classical version of the QFT can be more quantum resource efficient \cite{griffiths1996semiclassicalQFT}.
\end{itemize}

%%%%%%%%%%%%%%%%%%%%%%%%%%%%%%%%%%%%%%%%%%%%%%%%%%%%%%%%%%%%%%%%%%%%%%

\subsubsection*{Example use cases}

\begin{itemize}
    \item Even though the QFT does not speedup the DFT, QFT is used as a subroutine in more involved quantum routines with large quantum speedup. Examples include quantum algorithms for the discrete logarithm problem, the hidden subgroup problem, the factoring problem, to name a few. QFT can be seen as the crucial quantum ingredient that allows for a super-polynomial end-to-end quantum speedup for these problems. We discuss this in the context of \hyperref[appl:cryptanalysis]{quantum cryptanalysis}. 
    \item The QFT appears in the standard circuit for \hyperref[prim:QPE]{quantum phase estimation}, where it is used to convert accrued phase estimation into a binary value that can be read out.
    \item The QFT is used for switching between the position and momentum bases in grid-based simulations of \hyperref[appl:ElectronicStructure]{quantum chemistry}~\cite{kassal2008QuantumSimChemicalDynamics} or \hyperref[appl:QuantumFieldTheories]{quantum field theories}~\cite{jordan2012QuantumFieldTheory}.
\end{itemize}

%%%%%%%%%%%%%%%%%%%%%%%%%%%%%%%%%%%%%%%%%%%%%%%%%%%%%%%%%%%%%%%%%%%%%%

\subsubsection*{Further reading}

\begin{itemize}
    \item Textbook reference \cite[Chapter 5]{nielsen2002QCQI}
    \item Wikipedia article \hyperlink{https://en.wikipedia.org/wiki/Quantum_Fourier_transform}{\it Quantum Fourier transform}
    \item The quantum Fourier transform can be generalized to other groups. The version presented above is for the group $\mathbb{Z}/(2^n\mathbb{Z})$. Its implementation for other abelian groups as well as non-abelian groups is discussed in \cite{childs2010QAlgosForAlgebraicProblems} and the references therein. 
\end{itemize}
%%
\printbibliography[heading=secbib,segment=\therefsegment]

\end{refsection}
