\section{Conclusions and Outlook\label{sec:conclusion_outlook}}

This work can be summarized as a perspective on quantum computing in the field of materials science. Rather than an exhaustive review of the existing literature, this paper aims at pointing out the challenges for a quantum-centric supercomputing architecture to achieve quantum advantage for materials science use cases. For that, we have presented a comprehensive view of topics at different level of abstraction, ranging from hybrid quantum-classical architectural problems to the the final use cases. We believe this is the major contribution of this paper.

The different sections have highlighted the challenges at various levels. The first two sections closely tie together, listing fundamental algorithms for simulation of quantum systems and specific hurdles that arise in implementing them at scale both for noisy quantum devices and fault-tolerant architectures, with an emphasis on the classical workloads. Then, we moved our focus closer to the architecture design arguing in details about the difficulties that are faced when optimizing quantum and classical workloads coming from hybrid algorithms. Before tying up all the threads in the section about use cases, we have discussed how classical simulations of quantum systems can be used both for approximate verification and to identify hard use cases. Finally, we have provided criteria for the identification of good use cases in materials science, drawing upon the previous sections. Based on these criteria, we have discussed a few exemplary cases, representative of the variety of applications of materials science in mind.   

One major takeaway from this perspective is to suggest how we can lay the grounds to think about quantum computing to work in synergy with classical high-performance computing in quantum-centric supercomputing centers. Materials science provides a great setting for use cases, which have potential for quantum advantage for scientific and industrial applications. 